%% It is just an% Latex template for MA305 Project Report, Spring 2019
%%%%%%%%%%%%%%%%%%%%%%%%%%%%%%%%%%%%%%%%%%%%%%%%%%%%%%%%%%%
\documentclass[11pt]{article}
\usepackage{graphicx}
\usepackage[pdftex]{color}
\usepackage{multicol}
\newcommand{\cred} {\textcolor{red}}
\usepackage{fancyhdr}
\newcommand{\horrule}[1]{\rule{\linewidth}{#1}}      % Horizontal rule 
\usepackage{listings}
\definecolor{codegreen}{rgb}{0,0.6,0}
\definecolor{codegray}{rgb}{0.5,0.5,0.5}
\definecolor{codepurple}{rgb}{0.58,0,0.82}
\definecolor{backcolour}{rgb}{0.95,0.95,0.92}
\lstdefinestyle{mystyle}{
backgroundcolor=\color{backcolour},   
commentstyle=\color{codegreen},
keywordstyle=\color{magenta},
numberstyle=\tiny\color{codegray},
stringstyle=\color{codepurple},
basicstyle=\footnotesize,
breakatwhitespace=false,         
breaklines=true,                 
captionpos=b,                    
keepspaces=true,                 
numbers=left,                    
numbersep=5pt,                  
showspaces=false,                
showstringspaces=false,
showtabs=false,                  
tabsize=2
}
\lstset{style=mystyle}
\begin{document}
%%%%%%%%%% TITLE PAGE %%%%%%%%%%
\begin{center}
{\it MA448, Fall 2019  \hfill Embry-Riddle Aeronautical University
 }\\
\horrule{0.5pt} \\[0.4cm]
{\bf \Large  % change this
Project 1: Numerical Solution of ODEs
}\\
\horrule{2pt} \\[5cm]
%%%%%%%%%%%%%%%%%%%%
Brian, David, Jack Nguyen % change this
\\[0.4cm]
\today % change this
\end{center}
\thispagestyle{empty}
\newpage
\begin{abstract}
\end{abstract}
\tableofcontents 
\newpage
%%%%%%%%%%  %%%%%%%%%%
\section{Introduction}\label{S:1}
%The text of this section.
It is a common occurrence in the natural sciences and engineering to encounter
differential equations which either cannot be solved analytically, or are too 
complex and time-consuming to attempt. For practical purposes, an approximation
of the solution is often sufficient to meet the demands of the problem, and a
multitude of algorithms and techniques have been developed to approximate 
ordinary differential equations (ODEs). This paper aims to evaluate these techniques by 
comparing them both to each other and to results published by professional mathematicians, 
and to educate the reader by solving a practical problem.
\section{Problem Statement}\label{S:2}
%The text of this section. 
State fully and precisely the mathematical problem.  
Explain meaning of all symbols used. Make clear what is given and what we are 
looking for. 
\subsection{Part I}\label{S:2.1}
%
In this section, a multitude of different methods for solving ODEs are compared 
to one another by solving both a stiff and non-stiff equation. A stiff equation
is one in which the step size of the numerical methods used must be drastically
changed over the domain of the solution to maintain absolute stability. The 
methods evaluated in this way are the explicit implicit Euler's Method, the
implicit Euler's Method, the Trapezoidal Method, the Classical Runge-Kutta 
fourth-order Method, the Fourth-order Adams-Bashforth-Moulton Method, and MATLAB's
builtin ODE solver ode45. These methods are evaluated by comparing the relative
errors of each solution relative to ode45 (a thoroughly tested algorithm), and 
the time required to solve each method.
\subsection{Part II}\label{S:2.2}
%
In the second part of this paper, a numerical method is developed for solving the
system of differential equations presented in the paper “Laura and Petrarch: An 
Intriguing Case of Cyclical Love Dynamics”. This paper aimed to develop a mathematical
model to simulate the emotional and inspirational cycle of the Italian poet Petrarch
and his love for Laura. The efficacy of the solver developed for this problem is
analysed by comparing the results to those in the paper.
\subsection{Part III}\label{S:2.3}
%
This paper concludes with simulating a pertinent real-world problem: that of the 
Lorenz problem, which arises in the study of dynamical systems. The Lorenz problem
is comprised of a series of three autonomous ODEs, and is noteworthy because of
the famous Lorenz attractor, which was discovered by analyzing this type of 
problem. The problem with oe(???) particular set of initial conditions and coefficients
is solved using the fourth-order Adams-Bashforth-Moulton Method.
\section{Method/Analysis}\label{S:3}
%Text introducing this section
Begin with naming or characterizing the method/approach to be used, perhaps explain
the basic idea behind it, to what type of problems it applies, under what 
conditions, what it achieves, what are its main features, advantages, 
disadvantages. Justify why it is applicable to this problem, stating clearly any 
assumptions you need to make about the problem for the method to apply. Name some 
other methods/approaches one could use, and if/why your method may be preferable.
\subsection{Part I}
%
Text introducing this subsection. 
\subsection{Part II}
%
Text introducing this subsubsection. 
\subsection{Part III}
%
Text introducing this subsubsection. 
\section{Solutions/Results}\label{S:4}
%Text introducing this section
This section contains the presentation of your solution and results.
Describe your implementation of the method(s) for this specific problem, any 
special features, numerical methods implementation  strategy, choices of any 
parameters, stopping criteria, etc.
Present the results in words and plots (annotate by hand if necessary), explain 
what they mean. Include your code in an Appendix. 
\subsection{A subsection}
%
Text introducing this subsection. 
\subsubsection{A subsubsection}
%
Text introducing this subsubsection. 
\subsubsection{A further subdivision}
%
Text introducing this subsubsection. 
\section{Discussion/Conclusions}\label{S:5}
%Text introducing this subsection
Interpret your solution physically, what we learn from it, comment on strengths and
weaknesses of the solution method, any nice features you want to brag about, 
possible ways to improve it (e.g. how to make it more accurate, more efficient), as
appropriate.
\begin{thebibliography}{100}
%List the materials used in the project. e.g., books, papers, web resources, codes, etc. 
\bibitem{a1}  
Heath, Michael T., Scientific Computing: An Introductory Survey, McGraw Hill, 2002.
%
%\bibitem{a2}
%
%\bibitem{a3} 
\end{thebibliography}
%\end{document}
%%%%%%%%%%%%%%%%%%%%%%%%%%%%%% section Appendix %%%%%%%%%%%%%%%%%%%%%
\newpage
\appendix 
\setcounter{section}{0}           
\section{Codes}\label{S:A}
%
Text introducing this appendix. Subsections and further divisions can also be used 
in appendices. 
\begin{lstlisting}[language=Matlab]
%Brian Danaher
%Group 2, problem 1a

clear;
clc;

%Initial Setup
y0 = 0;
t0 = 0;
n = 65;
tmax = 10;
h = (tmax-t0)/n;
explicit_euler = y0;
implicit_euler = y0;
trapezoidal = y0;
RK4 = y0;
AB4 = y0;
matlab = y0;
t = t0;
tmat = linspace(t, tmax, 65);

%Function definition
fy = @(t,y) 3+5*sin(t)+0.2*y;

tic
%Explicit Euler Method
for a=1:1:n-1
    explicit_euler = [explicit_euler, explicit_euler(1,a)+(h*fy(t, explicit_euler(1,a)))];
    t = t+h;
   
end
explicit_euler_time = toc;

%Implicit Euler Method
t = t0;

tic
for b=1:1:n-1
    t = t+h;
    implicit_euler = [implicit_euler, (implicit_euler(1,b)+3*h+5*h*sin(t))/(1-0.2*h)];
    
end
implicit_euler_time = toc;

%Trapezoidal Method
t = t0;

tic
for c=1:1:n-1
    t1 = t;
    t2 = t+h;
    
    trapezoidal = [trapezoidal, (trapezoidal(1,c)+3*h+(5*h/2)*sin(t1)+(h/10)*trapezoidal(1,c)+(5*h/2)*sin(t2))/(1-h/10)];
    t = t+h;
    
end
trapezoidal_time = toc;

%4th-order Classical RK Method
t = t0;

tic
for d=1:1:n-1   
    k1y = fy(t, RK4(1,d));
    k2y = fy(t+(h/2), RK4(1,d)+(k1y*(h/2)));
    k3y = fy(t+(h/2), RK4(1,d)+(k2y*(h/2)));
    k4y = fy(t+h, RK4(1,d)+k3y*h);
    
    RK4 = [RK4, RK4(1,d)+(h*(k1y+2*k2y+2*k3y+k4y)/6)];
    t = t+h;
    
end
RK4_time = toc;
    
%4th-order Adams Bashforth Method

%We will use the Explicit Euler Method calculated prior to start
abmat = [explicit_euler(1,1), explicit_euler(1,2), explicit_euler(1,3), explicit_euler(1,4)];
t = t0+(4*h);

tic
for e=4:1:n-1  
    ybar = abmat(1,e)+(h/24)*(55*fy(t, abmat(1,e))-59*fy(t-h, abmat(1,e-1))+37*fy(t-2*h, abmat(1,e-2))-9*fy(t-3*h, abmat(1,e-3)));
    fbar = fy(t+h, ybar);
    abmat = [abmat, abmat(1,e)+(h/24)*(9*fbar+19*fy(t, abmat(1,e))-5*fy(t-h, abmat(1,e-1))+fy(t-2*h, abmat(1,e-2)))];
    t=t+h;
    
end
ABM_time = toc;

%ode45
tic
[ode45t, ode45y] = ode45(fy, [0,10], 0);
ode45_time = toc;

%Plot the data
plot(tmat, explicit_euler, '-o')
hold on
plot(tmat, RK4, '-x')
plot(ode45t, ode45y, '-*')
plot(tmat, abmat, '-o')
plot(tmat, implicit_euler, '-o')
plot(tmat, trapezoidal, '-o')

legend( 'Explicit Euler', 'RK4', 'ode45', 'Adams-Bashforth-Moulton', 'Implicit Euler', 'Trapezoidal')
\end{lstlisting} 
\end{document}
%% Write your code here.