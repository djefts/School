\documentclass[11pt]{article}     

\usepackage{fullpage}

\renewcommand{\baselinestretch}{1.0}

\newcommand\tab[1][1cm]{\hspace*{#1}}

\usepackage{amsmath,amsthm,amssymb,amsfonts} % Typical maths resource packages
%\usepackage{graphicx}                 % Packages to allow inclusion of graphics
%\usepackage{color}                    % For creating coloured text and background
%\usepackage{multicol}
%\usepackage[all]{xy}

\oddsidemargin 0cm
\evensidemargin 0cm

%\newtheorem{theorem}{Theorem}[section]
%\newtheorem{proposition}[theorem]{Proposition} 
%\newtheorem{corollary}[theorem]{Corollary}
%\newtheorem{lemma}[theorem]{Lemma}
%\newtheorem{remark}[theorem]{Remark}
%\newtheorem{definition}[theorem]{Definition}


% ******* User defined commands  *****************


\def\R{\mathbb{ R}}
\def\S{\mathbb{ S}}

\begin{document}
{\Large {\bf CS 222 Homework 4   \tab\tab\tab\tab DAVID JEFTS}}\\

\noindent \textbf{Online Submission via Canvas Only!} If you are not able to produce a PDF version, you can scan or take picture of your homework for submission. No paper submission will be accepted.\\

\noindent \textbf{Write your name on this sheet}. \textbf{No name or cover sheet will miss 2 points}

\begin{enumerate}
  \item (25 pts) Prove by induction: {\bf P}: $\forall n \in Z^+$, $(2c+1)^n$ is an odd number, where $c \in Z$ \\
    \tab P(n): $(2c+1)^n$ \\
    \tab \textbf{Base Step -} $P(n=1) = (2c+1)^1 = 2c+1 \Rightarrow$ this is an odd number\\
    \tab \textbf{Induction Step -} Assume by Induction Hypothesis \textbf{P(n)} holds for all n.\\
    \tab \textbf{Prove -}  P(n+1) \\
    \tab\tab $(2c+1)^{n+1} = (2c+1)^n\times(2c+1)^1 $ \\
    \tab\tab $= {odd number} \times {odd number} $ will always produce an odd number. \\
    \tab\tab\tab$\therefore$ P(n) holds by induction. {\Large\checkmark} \\
    
  \item (25 pts) Prove by induction: {\bf P}: $\sum_{i=1}^n\frac{1}{2^i}=1-\frac{1}{2^n} $ \\
  \tab P(n): $\sum_{i=1}^{n}\frac{1}{2^i} = 1-\frac{1}{2^n} $ \\
  \tab \textbf{Basis Step -} P(n=1) $= \frac{1}{2^1} = 1 - \frac{1}{2^1} \Rightarrow 1=1 $ \\
  \tab \textbf{Induction Step -} Assume by Induction Hypothesis \textbf{P(n)} holds for all n. \\
  \tab \textbf{Prove -} P(n+1) \\
  \tab\tab $ \sum_{i=1}^{n}\frac{1}{2^i}+\frac{1}{2^{n+1}} = 1 - \frac{1}{2^{n+1}} \tab\tab\tab\tab\tab \sum_{i=1}^{n}\frac{1}{2^i} = 1-\frac{1}{2^n} $ \\\\
  \tab\tab $ 1-\frac{1}{2^n} + \frac{1}{2^{n+1}} = 1 - \frac{1}{2^{n+1}} $ \\\\
  \tab\tab $ 1-\frac{1}{2^n}\times\frac{2}{2} + \frac{1}{2^{n+1}} = 1 - \frac{1}{2^{n+1}} \Rightarrow 1-\frac{1}{2^{n+1}} + \frac{2}{2^{n+1}} = 1 - \frac{1}{2^{n+1}} $ \\\\
  \tab\tab $ 1+\frac{-1+2}{2^{n+1}} = 1 - \frac{1}{2^{n+1}} \Rightarrow 1 - \frac{1}{2^{n+1}}=1 - \frac{1}{2^{n+1}} $ \\
  \tab\tab\tab$\therefore$ P(n) holds by induction. {\Large\checkmark}\\\\
  

  \item (25 pts) Prove by induction: {\bf P}: $\forall n\in Z^+$, $8^n-3^n \equiv 5k$ \\
    \tab P(n): $8^n-3^n$ is divisible by 5. \\
    \tab \textbf{Base Step -} $ P(n=1) = 8^1-3^1 = 8-3=5 \equiv 5k\tab{\Large\checkmark}$ \\
    \tab \textbf{Induction Step -} Assume by Induction Hypothesis \textbf{P(n)} holds for all n.  \\
    \tab \textbf{Prove -}  P(n+1) \\
    \tab\tab $ 8^{n+1}-3^{n+1} = 8^n\times8-3^n\times3 = (5+3)\times8^n - 3\times3^n$ \\
    \tab\tab $ 5 + 3(8^n-3^n) = 5 +3(5k) \equiv 5k$ \tab\tab\tab\tab$8^n-3^n \equiv 5k $ \\
    \tab\tab\tab $\therefore$ P(n) holds by induction. {\Large \checkmark}\\\\\\\\
  

  \item (25 pts) Use \textbf{strong induction} to show that you can run any number of miles given the following two facts:	
  \begin{itemize}
    \item You can run one mile or two miles. 
    \item You can always run two more miles once you have run a specified number of miles \\\\
    $P(k): $ \\
    $P(k): k=1m+2n$ $\Rightarrow$ $k=\#$ of miles, $m=1$ mile and $n=2$ miles \\
    \textbf{Basis Step -} \\
    \tab$P(k=1): k=1, m=1$ and $n=0 \rightarrow 1=1(1)+2(0) = 1 \checkmark$ \\
    \tab$P(k=2): k=2, m=0$ and $n=1 \rightarrow 2=1(0)+2(1) = 2 \checkmark$ \\
    \textbf{Induction Step -} \\
    \tab Assume by induction hypothesis that $P(1)$ to $P(k)$ is true.\\
    \tab\textbf{Prove P(k+1):} \\
    \tab\tab Induction \\
    \tab\tab\tab$(k+1) > 2$ \\
    \tab\tab\tab$(k+1)-2 > 2-2=0$ \\
    \tab\tab\tab$(k+1)-2 > 0$ \\
    \tab\tab \textbf{Proof:} \\
    \tab\tab\tab $(k+1)-2$ is within the range of $P(1)$ to $P(k)$\\\\
    \tab\tab\tab$(k+1)-2 \leq 1m+2n$ \\
    \tab\tab\tab$(k+1)-2 \leq k$ \\
    \tab\tab\tab$k-1 \leq k \checkmark$\\
        
    \end{itemize}


\end{enumerate}

\end {document}

