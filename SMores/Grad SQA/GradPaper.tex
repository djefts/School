\documentclass[10pt, compsoc]{IEEEtran}
\usepackage[utf8]{inputenc}
\usepackage{indentfirst}
\usepackage{graphicx}
\usepackage{setspace}
\usepackage[numbers]{natbib}
\usepackage{layout}

\begin{document}

\title{Assuring Software Quality Through the Use of Fuzzy Testing}
\author{David Jefts, \IEEEmembership{SE625, Embry-Riddle Aeronautical University}}

\maketitle

%%%%%%
%I assume you will introduce the concept of Fuzzy testing, how it is used, where it is used, advantages disadvantages, etc. etc.   If this is correct, you are good to go.
%%%%%%

\begin{IEEEkeywords}
	Fuzzy Testing, Software Quality Assurance, Software Vulnerabilities
\end{IEEEkeywords}

\section{Introduction}
	Reason I chose this topic
	
	Goal of this paper


\section{Main Body}
	What is fuzzy testing?
	
	Advantages and disadvantages of fuzzy testing
	
	Fuzzy testing techniques
	
	Examples of use of fuzzy testing in industry
	
	Fuzzy Testing vs Other Fuzzy Testing techniques
	
	\indent\indent Explain the context of these comparisons
		
	Analyze potential applications and best-case uses for fuzzy testing


\section{Conclusion}
	Is fuzzy testing better or worse than other methods of testing and for which cases is it better?

\pagebreak
	
%Appendices
\appendix
	\section{Appendix A}
	
	\section{Appendix B}
	
%Bibliography
\newpage
\nocite{*}
\bibliographystyle{apa}
\bibliography{bibliography.bib}

\end{document}
